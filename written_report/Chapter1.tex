\chapter{Introduction}\label{cha1}
\section{Kite Power in General and the SwissKitePower Project}
At a time when windmills were already quite commonly used for power generation, Miles Loyd came up with the idea to use kites to convert wind energy to electricity.  In 1980 he wrote a seminal paper exploring the possibility of generating electrical power using the pulling force of tethered airfoils, i.e., kites \cite{loyd1980}. He describes his concept as follows:
\begin{quote} A kite's aerodynamic surface converts wind energy into motion of the kite. This motion may be converted into useful
power by driving turbines on the kite or by pulling a load on the ground. [...] Not simply facing into the wind, such kites would fly a closed path downwind from the tether point. The kite's motion would be approximately transverse to the wind, in the same sense that a wind turbine's blade moves transverse to the wind. The crosswind airspeed of a kite with this trajectory is increased above the wind speed by the lift-to-drag ratio L/D. The resultant aerodynamic lift is sufficient to support a kite and to generate power. \cite{loyd1980}\end{quote}
Today, several research groups around the world are investigating this subject and working on prototypes. For example the University of Torino have already tested a prototype \cite{canale2006} and Massimo Ippolito has founded a company named KiteGen that is also located in Torino. At the University of Delft the group of Prof. Dr. Wubbo Oeckels is developing kites to produce energy with "Laddermills" \cite{schmehl2012} and at the K. U. Leuven the group of Prof. Moritz Diehl is working on the Highwind project \cite{geebelen2012}. Furthermore the SkySails company is already using wind power to pull large cargo ships \cite{erhard2012} and Ampyx Power, a spin-off from T.U.Delft is working on their PowerPlane device.

In Switzerland the research efforts are coordinated in the SwissKitePower Project \cite{skp}. SwissKitePower is a collaborative research and development project between FHNW (University of Applied Sciences Northwestern Switzerland), ETHZ (Swiss Federal Institute of Technology Zurich), EMPA (Swiss Federal Laboratories for Materials Testing and Research) and Alstom Switzerland AG. The goals of the project are to develop 'novel wind energy extraction technology' using tethered airfoils, or kites, and to promote this innovative new technology to the world.

\section{Related Work}
There are many publications on the topic of IMU/GPS data fusion for navigation purposes. As a starting point of our work we used the Master Thesis of Andr\'{a}s H\'{e}jj "Kalman-filter based position and attitude estimation algorithms for an Inertial Measurement Unit" \cite{hejj}. In an other Master project at the University of Southern Denmark Ushanthan Jeyabalan implemented a Kalman Filter using a spherical pendulum model \cite{ushanthan}.
Several other publications investigated the use of a Kalman Filter for IMU/GPS data fusion in different applications. Sukkarieh applied it to land vehicles \cite{sukkarieh1999}, Kim used it for unmanned areal vehicles in highly dynamic flight situations \cite{kim2006} and Crassidis used a Sigma Point Kalman Filter and compared it to an EKF. This list of course is not exhaustive. A much broader overview over the various approaches to multisensor data fusion in target tracking can be found in the survey article of Smith and Singh \cite{smith2006}.
 


\section{Motivation and Methods}
To successfully implement an automated control algorithm on a kite, it is essential that a precise and fast position estimation is available. Due to the slow update rate of the GPS units and their limited reliability, a Kalman Filter will be used to integrate IMU measurements, which consist of acceleration, rate of turn and magnetic field, with the GPS position and velocity estimations. A standard Kalman Filter, one that could also be used in an airplane for example, assumes no knowledge aboout forces acting on the body. Such an estimator will already enhance the position estimation from the raw GPS input since it incorporates more measurements.  However, a kite can only move on a restricted surface due to it being thethered to the groundstation. An extended Kalman Filter could take advantage of that knowledge and further improve the estimator's performance. Further integrating an aerodynamical model of the kite could give more information about the forces acting on it and would reduce the estimator's dependency on precise sensor output.  

In order to be able to compare the performance of the different estimators, it is neccessary to know the exact location and orientation of the kite at all times. In practice it is nearly impossible to obtain such a ground truth for a kite flying around in the air. Therefore we decided to investigate the benefits of an accurate physical model on the performance of a Kalman Filter using box suspended by a string. (See figure ...) This setup can be modeled as a spherical pendulum which has the advantage that all the forces acting on it are known.  In addition, we are able to test the algorithms indoors using a Vicon System (add reference!!) that gives us a very accurate ground truth of the box's position and orientation. Since there is obviously no GPS reception indoors, the GPS output needs to be simulated using the Vicon's position output, reducing its frequency and adding an appropriate level of noise.



