\chapter{Conclusion and Outlook}\label{cha}
\section{Sensor Selection}
The centrifuge test was not able to give a meaningful conclusion about the quality of the different sensors. There seems to be a correlation between the acceleration acting on the sensors and their accuracy, but this could also be due to more vibrations at higher speeds. However, the test showed some general limitations in the range of the different IMU's. The PX4 was not able to cover the whole acceleration range and the MTi-G could not measure rates of turn larger than 403 deg/s. In order to be able to better compare these different units, a test would have to be done where a precise ground truth would be available. Furthermore, the MTi-G should be set to output the calibrated measurements, as this spares the conversion to conventional units.

\section{State Estimation}
Based on the Vicon test, we were able to show an improvement of accuracy in the state estimation when using the "pendulum model"-estimator. Especially in the absence of a GPS signal, this algorithm was still able to estimate the true position while the conventional Kalman Filter drifted off during such periods. This improvement is on one hand due to the reduction in degrees of freedom, giving more redundant measurements per state, and on the other hand due to the accurate prediction of the future states by using the exact knwoledge of the forces acting on the body. These promising results make us confident, that position estimation of a kite with an Extended Kalman Filter could be improved by including as much knowledge of the system as possible. The complete independence of the GPS measurements in the pendulum case, suggests that also in the more complex system of the kite, the dependency on an accurate GPS signal could be reduced.

For future work in this subject, we suggest to also investigate the use of models with 8 or more degrees of freedom, allowing for example the radius to vary. Furthermore, the ideal combination of Covariance matrices would have to be found either by finding the exact uncertainties in the sensors and the prediction, or by minimizing the output error. Considering the application of the algorithm to kites, it would be interesting to find out how sensitive the filter is to errors in the prediction of the forces acting on the body. If an very accurate propagation model, as the one of the pendulm that was used in this work, is needed, the algorithms application to a real kite system could prove to be difficult since these systems are a lot more complex. However, investigating the use of additional sensors, like for example line angle sensors, the possibilities of an exact estimation of the kites position and orientation without relying on a GPS signal could be determined. 
